\documentclass[10pt, a4paper]{article}
\usepackage[margin=2.0cm]{geometry}
\usepackage{stmaryrd}
\usepackage{amsmath}
\usepackage{amssymb}
\usepackage{enumerate}
\usepackage{bbold}
\usepackage{dsfont}
\usepackage{wrapfig}
\usepackage{algorithm2e}
\usepackage{changepage}
\usepackage{undertilde}
\usepackage{hyperref}
\hypersetup{pdftex,colorlinks=true,allcolors=blue}
\usepackage{hypcap}
\usepackage{multicol}
\usepackage{float}
\usepackage{tikz}
\usepackage{qtree}

\usepackage[amsmath,hyperref]{ntheorem}
\usepackage{framed}
\usepackage{booktabs}
\usepackage{needspace}

\newcommand{\thm}{
    \theoremstyle{plain}
    \theoremseparator{.}
    \theorembodyfont{\upshape}
    \theoremheaderfont{\bfseries}
}

\thm\newtheorem{definition}{Definition}[subsection]
\thm\newtheorem{proposition}[definition]{Proposition}
\thm\newtheorem{theorem}[definition]{Theorem}
\thm\newtheorem{lemma}[definition]{Lemma}


\setlength{\parskip}{1em}
\setlength{\parindent}{0em}
\SetKwProg{Fn}{Function}{\string:}{}

% Default fixed font does not support bold face
\DeclareFixedFont{\ttb}{T1}{txtt}{bx}{n}{10} % for bold
\DeclareFixedFont{\ttm}{T1}{txtt}{m}{n}{10}  % for normal

% Custom colors
\usepackage{color}
\definecolor{deepblue}{rgb}{0,0,0.5}
\definecolor{deepred}{rgb}{0.6,0,0}
\definecolor{deepgreen}{rgb}{0,0.5,0}

\usepackage{listings}

% Python style for highlighting
\newcommand\pythonstyle{\lstset{
language=Python,
basicstyle=\ttm,
otherkeywords={self},             % Add keywords here
keywordstyle=\ttb\color{deepblue},
emph={MyClass,__init__},          % Custom highlighting
emphstyle=\ttb\color{deepred},    % Custom highlighting style
stringstyle=\color{deepgreen},
frame=tb,                         % Any extra options here
showstringspaces=false            %
}}


% Python environment
\lstnewenvironment{python}[1][]
{
\pythonstyle
\lstset{#1}
}
{}

% Python for external files
\newcommand\pythonexternal[2][]{{
\pythonstyle
\lstinputlisting[#1]{#2}}}

% Python for inline
\newcommand\pythoninline[1]{{\pythonstyle\lstinline!#1!}}
\lstset{showstringspaces=false}

\newcommand\Var{\text{Var}}
\renewcommand\vec[1]{\utilde{#1}}

\newcommand{\pic}[2]{\begin{figure}[!ht]
    \centering
    \includegraphics[width=\textwidth]{#1}
    \caption{#2}
 \end{figure}
}

\begin{document}

	\begin{center}
	{\Large\bf Image Processing and Computer Vision}

	\vspace{.1in}

	Course Summary

  Dylan Cope

	\vspace{.2in}

	\end{center}

	\section{Modelling and Image}

    For coordinates $x_1, x_2, \dots, x_m$ and colour values $c_1, c_2, \dots, c_n$, an image encodes a function $f : \mathcal{R}^m \rightarrow \mathcal{R}^n$ mapping between these variables, $m$ is the image dimensionality and $n$ is the number of colour channels.

  \section{Convolutions}

  \section{Hough Transforms}

  Given an image $\underset{m \times n}{I}$ and a shape described by $p$ parameters, the Hough transform $H$ is a $p$-dimensional space that maps every possible instance of the given shape to a value indicating its presence in $I$. The process of detecting shapes in images using Hough transforms is to find shapes that have values in the transform greater than some threshold.
  

  \subsection{Line Detecting}

  Lines are characteristic of two parameters; the $r$ and $\varphi$ values of the equation $r = x\sin \varphi + y\cos \varphi$. This means the Hough transform for a given image is two-dimensional.

  \subsection{Circle Detecting}

  \subsection{Generalised Hough Transform}

  \section{Viola-Jones Object Detection}

  \subsection{Integral Image}

  The integral image $\underset{m \times n}{II}$ is a transformation of an image $\underset{m \times n}{I}$ where each entry defined by the following recursive relationship,

  $$
    ii(x, y) = \begin{cases}
      0 & \text{if } x \notin [0, n) \text{ or } y \notin [0, m) \\
      I_{x, y} + ii(x - 1, y) + ii(x, y - 1) - ii(x-1, y-1) & \text{otherwise}
    \end{cases}
  $$

  Through memoization, $II$ is computed in $O(mn)$ time (and $II$ is the resultant memoization table).

  \subsection{Haar-like Features}

  \subsection{AdaBoosting}

  \subsection{Performance}

  \section{Motion}

  \subsection{Lucas and Kanade Algorithm}

  \subsection{Aperature Problem}

  \subsection{Segmentation by Velocity}

  \section{Stereo Vision}

  \subsection{Calibration}

  \subsection{Correspondance Problem}

\end{document}
